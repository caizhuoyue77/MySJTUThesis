% !TEX root = ../main.tex

\chapter{绪论}

\section{本文研究内容}
本文主要研究基于Agent与图谱的任务编排工具的设计与实现流程,主要针对用户进行信息查询时的便利。本文研究与实现的内容主要包括以下几点:

\begin{enumerate}
    \item 研究了API过程数据的采集和API知识图谱的搭建。【需要补充】
    \item 研究了基于知识图谱的DFS动态搜索算法。【需要补充】
    \item 基于上述算法和机制,设计并构建了基于Agent与图谱的任务编排的具体流程。该流程以用户与系统对话的方式建立,可以通过自然语言输入用户的需求,并通过任务分解、图谱搜索和API编排等环节最终得到API调用链,并调用API得到结果。
    \item 设计并实现了系统的各个功能模块,包括知识采集与图谱搭建、基于知识图谱的DFS动态搜索、基于语义相似度的长期记忆搜索、用户自定义API编排流程等功能,以及这些功能所对应的可视化系统界面。
\end{enumerate}

\section{国内外研究现状}
本节将会概述针对大语言模型智能体和知识图谱的研究进展。首先,本节会介绍大语言模型智能体工具调用方面的研究,分为基于提示词工程的方法和基于模型微调的方法。其次,本节将聚焦于大语言模型规划、推理提升的方面,包括提示词工程和模型微调的方法。最后,本节还会介绍知识图谱和大语言模型结合的应用,比如如何利用知识图谱中的外部知识增强大语言模型,以及如何在图谱上进行推理。

\subsection{大语言模型智能体工具调用}
外部工具的引入不仅能够增强大语言模型的能力,比如获取最新知识、提升专业技能,流程自动化、交互增强等。同时,外部工具的采用也能够提高生成过程中的透明性和鲁棒性,让回答更加可靠和可解释。在大语言模型智能体工具调用领域,研究人员已经提出了许多方法来实现大语言模型工具调用。总体来讲,许多工作中的大语言模型工具调用流程都包含以下四个阶段:任务规划、工具选择、工具调用和响应生成\cite{Qin2023,Shen2023,Ruan2023}。

本文主要聚焦于任务规划和工具选择的部分,即如何根据用户需求从众多工具中选择合适的一个或一组工具来形成工具调用链来完成用户的需求。

\subsubsection{工具任务规划}
在现实的信息查询场景中,用户的查询需求往往包含复杂的意图,如何识别用户意图是工具调用的首要问题。因此在工具任务规划阶段,我们首先需要将用户的需求语句转化为更加明确的任务,进行子任务的拆解和任务之间的关联分析。任务规划方式一般分为基于提示词工程的方式和基于微调的方式。

\paragraph{基于提示词工程的工具规划} 
现有研究\cite{Miao2023}表示,大语言模型能够通过少样本甚至零样本实现有效的任务规划。HuggingGPT\cite{Shen2023}首先把任务分解为各种子任务,然后选择合适的模型来解决这些子任务。RestGPT\cite{Song2023}引入了一种从粗粒度到细粒度的规划方法,能够指导大语言模型逐步对任务进行分解。ControlLLM\cite{Liu2023}引入了一种叫做“在图上思考”(Think-on-graph, ToG)的范式,通过深度优先搜索算法(DFS)在构建好的工具图上进行搜索,得到解决方案。

\paragraph{基于微调的工具规划} 
Toolformer\cite{Schick2023}通过工具调用来辅助模型预测后续token,基于此原理对模型进行了微调,从而提升大语言模型的工具认知和工具调用效率。TookenGPT\cite{Hao2023}中将工具作为了特殊token,以生成普通文字输出的方式对外部工具进行调用。

\subsubsection{工具选择}
在任务规划阶段完成后,需要根据每个子任务进行工具选择。工具选择过程一般有两种途径:一种是通过训练得到的检索器来选择工具,另一种是直接让大语言模型从工具列表中选择合适的工具。

\paragraph{基于检索器的工具选择} 
当工具数量过多时,通常会使用检索器先搜索得到与任务相关的工具。检索方式包括基于关键词的检索和基于语义的检索两种。

\begin{enumerate}
    \item \textbf{基于关键词的检索}:如TF-IDF\cite{sparck1972statistical}和BM25\cite{Robertson2009}。这些方法通过精确匹配实现用户需求和文档之间的对齐和查询。
    \item \textbf{基于语义的检索}:利用神经网络来学习文本之间的语义关系,然后使用余弦相似度等算法计算语义相似度,如ToolLLM\cite{Qin2023}。
\end{enumerate}

\paragraph{基于大语言模型的工具选择} 
在工具数量有限时,可以让大语言模型根据输入上下文分析工具并做出选择。此类方法包括基于提示词工程的方法和基于模型微调的方法。

\subsection{大语言模型与知识图谱结合的应用}
尽管大语言模型主要用于纯文本的场景,但是在许多现实场景中,文本数据与结构化的信息以图谱的方式存储。通过将现实世界的知识表示为结构化的知识图谱,并在图谱上进行推理和演算,能够解决许多重要问题\cite{Liu2024}。

\section{论文组织结构}
本论文的内容组织结构分为以下几章:

\begin{itemize}
    \item 第一章为绪论,介绍了本研究的背景和研究意义。
    \item 第二章为相关理论与技术,介绍了知识图谱和大语言模型的发展及智能体技术。
    \item 第三章【需要补充】
    \item 第四章【需要补充】
    \item 第五章为系统设计和实现,介绍了系统的架构以及各功能模块的设计与实现。
    \item 第六章为总结与展望,对全文的研究工作进行回顾,并对未来的研究方向进行展望。
\end{itemize}

\section{本章小结}
本章概述了研究背景和本文的研究意义,介绍了国内外的研究现状与进展,以及本文的研究内容和贡献。最后,介绍了论文的组织结构。