% !TEX root = ../main.tex

\chapter{绪论}

\section{本文研究内容}
本文主要研究基于Agent与图谱的任务编排工具的设计与实现流程,主要针对用户进行信息查询时的便利。本文研究与实现的内容主要包括以下几点:

\begin{enumerate}
    \item 研究了API过程数据的采集和API知识图谱的搭建。【需要补充】
    \item 研究了基于知识图谱的DFS动态搜索算法。【需要补充】
    \item 基于上述算法和机制,设计并构建了基于Agent与图谱的任务编排的具体流程。该流程以用户与系统对话的方式建立,可以通过自然语言输入用户的需求,并通过任务分解、图谱搜索和API编排等环节最终得到API调用链,并调用API得到结果。
    \item 设计并实现了系统的各个功能模块,包括知识采集与图谱搭建、基于知识图谱的DFS动态搜索、基于语义相似度的长期记忆搜索、用户自定义API编排流程等功能,以及这些功能所对应的可视化系统界面。
\end{enumerate}

\section{国内外研究现状}
本节将会概述针对大语言模型智能体和知识图谱的研究进展。首先,本节会介绍大语言模型智能体工具调用方面的研究,分为基于提示词工程的方法和基于模型微调的方法。其次,本节将聚焦于大语言模型规划、推理提升的方面,包括提示词工程和模型微调的方法。最后,本节还会介绍知识图谱和大语言模型结合的应用,比如如何利用知识图谱中的外部知识增强大语言模型,以及如何在图谱上进行推理。

\subsection{大语言模型智能体工具调用}

外部工具的引入不仅能够增强大语言模型的能力,比如获取最新知识、提升专业技能,流程自动化、交互增强等。同时,外部工具的采用也能够提高生成过程中的透明性和鲁棒性,让回答更加可靠和可解释。在大语言模型智能体工具调用领域,研究人员已经提出了许多方法来实现大语言模型工具调用。总体来讲,许多工作中的大语言模型工具调用流程都包含以下四个阶段:任务规划、工具选择、工具调用和响应生成(ref)。

本文主要聚焦于任务规划和工具选择的部分,即如何根据用户需求从众多工具中选择合适的一个或一组工具来形成工具调用链来完成用户的需求。

\subsubsection{工具任务规划}
在现实的信息查询场景中,用户的查询需求往往包含复杂的意图,如何识别用户意图是工具调用的首要问题。因此在工具任务规划阶段,我们首先需要将用户的需求语句转化为更加明确的任务,进行子任务的拆解和任务之间的关联分析。任务规划方式一般分为基于提示词工程的方式和基于微调的方式。

\paragraph{基于提示词工程的工具规划} 
现有研究\cite{Miao2023}表示,大语言模型能够通过少样本甚至零样本实现有效的任务规划。HuggingGPT\cite{Shen2023}首先把任务分解为各种子任务,然后选择合适的模型来解决这些子任务。RestGPT\cite{Song2023}引入了一种从粗粒度到细粒度的规划方法,能够指导大语言模型逐步对任务进行分解。ControlLLM\cite{Liu2023}引入了一种叫做“在图上思考”(Think-on-graph, ToG)的范式,通过深度优先搜索算法(DFS)在构建好的工具图上进行搜索,得到解决方案。

\paragraph{基于微调的工具规划} 

Toolformer通过工具调用来辅助模型预测后续词元,基于此原理对模型进行了微调,从而提升大语言模型的工具认知和工具调用效率。TookenGPT中将工具作为了特殊词符,以生成普通文字输出的方式对外部工具进行调用。α-Umi提出了一种新的两阶段的训练模式,首先对基础大语言模型进行较为通用的微调,随后细分为规划器、调用器等模块,分别进行更加针对性的微调。

\subsubsection{工具选择}
在任务规划阶段完成后,需要根据每个子任务进行工具选择。工具选择过程一般有两种途径:一种是通过训练得到的检索器来选择工具,另一种是直接让大语言模型从工具列表中选择合适的工具。

\paragraph{基于检索器的工具选择} 
当工具数量过多时,通常会使用检索器先搜索得到与任务相关的工具。检索方式包括基于关键词的检索和基于语义的检索两种。

\begin{enumerate}
    \item \textbf{基于关键词的检索}:如TF-IDF\cite{sparck1972statistical}和BM25\cite{Robertson2009}。这些方法通过精确匹配实现用户需求和文档之间的对齐和查询。
    \item \textbf{基于语义的检索}:利用神经网络来学习文本之间的语义关系,然后使用余弦相似度等算法计算语义相似度,如ToolLLM\cite{Qin2023}。
\end{enumerate}

\paragraph{基于大语言模型的工具选择} 

在工具数量有限,或者是已经检索得到少量有关工具时,可以让大语言模型利用自身的推理和分析能力选择最合适的工具。具体来说,我们可以将备选工具的工具名称、工具描述信息和参数列表与用户需求一起放入大语言模型的输入上下文,提供给模型。随后,模型根据用户需求选择合适的工具。现有的基于大语言模型的工具调用方法分为两类:基于提示词工程的方法和基于模型微调的方法。

\textbf{基于提示词工程的方法}:该方法利用大语言模型的上下文学习能力,通过编写提示词来进行工作。有一些通用的提示词技巧可以帮助在多个工具中选择正确的工具。Chain of Thought(CoT)[14]在提示词中加入了例子,让大模型在解决复杂问题时采取相应的推理步骤,让大模型以分步的方式来规划和行动。Re-Prompting[15]在生成计划之前会检查每个步骤是否能够执行。如果不能够执行,则让大模型重新生成计划。Self-consistent CoT(CoT-SC)[16]因此让大模型执行多条推理路径,选择出现频率最高的答案输出。Tree of Thoughts(ToT)[17]用树状的形式组织推理过程,树上的每个节点表示一个“想法”即推理中间步骤。中间步骤的选择基于大模型的评估,最终计划用深度优先遍历(DFS)或者广度优先遍历(BFS)得出。在GoT[18]中,作者把用树状结构组织推理扩展为了用图结构组织。

引入环境反馈同样可以提升能力,ReAct[19]中指导大模型按照thought-action-observation的方式来解决问题。生成的想法来帮助大模型进行推理和规划,基于这个想法大模型会采取不同的行动,最后观察该行为的结果并作为反馈提供给大模型。Voyager[20]里智能体接收的反馈包括三种:程序执行的中间结果、执行错误描述和自我验证结果。Inner Monologue[21]主动获取人类的反馈,将其与环境反馈进行结合,用于增强大模型的规划和推理能力。SelfCheck[22]则让智能体对自己的推理步骤进行检查和评估,根据结果来修改计划以提升性能。

关于专门针对工具选择场景的提示词工程和流程搭建工作,ToolNet[23]将大量工具组织成为有向图的形式,允许大语言模型从初始节点出发,迭代地在图上选择下一个工具,直到完成任务。ToolLLM[11]中提出了基于深度优先遍历算法的决策树算法,通过支持回溯操作解决了在工具选择上的错误传播问题,有效提高了整体的准确性和通过率。AnyTool[24]提出了一种自我反思的层次化选择的方法,通过在结构化的工具调用树上迭代选择合适的工具。

\textbf{基于模型微调的方法}:该方法通过对大语言模型进行参数微调,以提高其在工具选择中的表现。此类方法通常涉及到利用额外的训练参数或者针对性训练,从而增强模型的工具选择能力。ToolLLaMA[11]利用在ToolBench数据集中DFSDT算法所得到的的指令-推理轨迹对微调了LLaMA- 7B的模型,有效增强了开源大模型的工具能力。ToolAlpaca[25]提出了一种自动化构建工具调用数据的框架,构建了3.9K条工具调用数据集微调得到了ToolAlpaca-7B和ToolAlpaca-13B的模型。ToolVerifier提出了一种“自我验证”的思想,通过在工具选择过程中自问自答一组问题来区分相似的候选工具。
不管是基于提示词工程还是基于模型微调的方法,都有各自的优缺点和特点,但是都能够有效提升模型的工具选择能力。基于提示词工程的方法不需要对模型参数进行修改,通过精心构建的提示词构建和流程搭建来提升大语言模型在工具选择中的能力,并适用于所有的大语言模型。而基于模型微调的方法相对复杂,并且仅能适用于开源大语言模型。在微调中需要消耗计算资源,通过调整参数的方式将工具有关的知识注入到模型中。


\subsection{大语言模型与知识图谱结合的应用}

尽管大语言模型主要用于纯文本的场景,但是在许多现实场景中,文本数据与丰富的结构信息以图谱的方式存储。此外,大语言模型的基于文本的推理能力已经得到较多的展现,但是大语言模型在图谱上的推理能力仍有很大的探索空间。

通过将现实世界的知识表示为结构化的知识图谱,并在图谱上进行推理和演算,能够解决许多重要问题。

关于如何将图谱上的知识提供给大语言模型的问题,有三种常见的方法:

\begin{enumerate}
    \item 自然语言描述。用自然语言描述图结构是最简单的方式,可以直接描述图上的边和邻接列表。
    \item 对图进行文本上的改写。由于自然语言描述图通常会较为复杂,而且不具备结构化的特点,因此对图的描述进行了改写,得到了更加高效的图的描述,有利于模型对图谱信息的利用。
    \item 对图进行编码。最后一种方法是通过训练图编码器,将图结构编码成为特征序列并作为特征的一部分输入到大语言模型中。这种方法涉及到对大语言模型的微调,以让其适应新的输入格式。
\end{enumerate}

通过将图谱的信息通过自然语言文本或者嵌入向量的格式输入到大语言模型上,可以在图谱上进行推理和搜索。常见的做法是利用深度优先搜索算法(DFS)或者广度优先搜索算法(BFS)来实现在图上的推理和搜索。许多研究探索了基于搜索的推理,特别是在知识图谱领域。Reasoning on Graph[26]的方法将知识图谱作为可靠的知识来源,通过提示大语言模型生成多个关系路径作为计划,随后根据这些路径不断在知识图谱上搜索,有效地提高了回答的可信度和效率。另一种方法是在图谱上动态地进行迭代检索和推理子图来模拟动态搜索过程[23,27,28]。在每个步骤中,大语言模型都检索当前节点的邻居节点,然后决定下一步操作是继续搜索还是结束搜索并给出答案。在ToolNet中[23],作者根据工具调用的数据集建立了图谱,并根据图谱上的边进行搜索,迭代式选择所需的工具进行调用,有效提升了工具搜索的准确性。Think-on-Graph系列[28,29]在知识图谱上通过大语言模型Agent进行迭代式的束搜索,探索发现最好的推理路径,并返回最有可能的推理结果。

这些方法的优点在于,通过图谱的辅助,系统不仅能够提供对应的答案,还可以提供图谱片段作为可以解释的证据。同时,由于知识图谱存储的数据量可以扩展,这些方法通过在图上搜索也增加了系统的可扩展性。


\section{论文组织结构}
本论文的内容组织结构分为以下几章:

\begin{itemize}
    \item 第一章为绪论,介绍了本研究的背景和研究意义。
    \item 第二章为相关理论与技术,介绍了知识图谱和大语言模型的发展及智能体技术。
    \item 第三章【需要补充】
    \item 第四章【需要补充】
    \item 第五章为系统设计和实现,介绍了系统的架构以及各功能模块的设计与实现。
    \item 第六章为总结与展望,对全文的研究工作进行回顾,并对未来的研究方向进行展望。
\end{itemize}

\section{本章小结}
本章概述了研究背景和本文的研究意义,介绍了国内外的研究现状与进展,以及本文的研究内容和贡献。最后,介绍了论文的组织结构。