% !TEX root = ../main.tex

\chapter{总结与展望}

\indent 本章共包括两个部分,第一个部分是对全文工作内容进行总结和回顾,第二个部分就是对本工作的不足和局限性进行探讨,并对未来可能的研究探索方向进行探讨。

\section{工作总结}

\indent 本文从xx。

本文工作主要包含以下三个部分的内容。

\section{未来展望}

本文围绕着如何构建有效的API工具图谱、并将大量API工具集成到大语言模型中进行了一系列的探索,
然而本文工作仍面临着许多不足之处和可以持续探索的空间。
以下几点是未来可以持续探索和改进的方向。

\indent 1.本文中提出的基于工具轨迹调用数据构建工具图谱的方法,
在工具图谱的构建过程中,我们只考虑了工具之间的跳转关系,而没有考虑到工具之间的替换关系。
然而在现实工具调用场景中,当一个工具不稳定或者实效时,用户可能
需要选择其他在功能上有替换关系的工具进行代替。
如何在工具图谱中考虑更多的工具之间的关系,并挖掘更丰富的工具关系,是
未来值得探索的方向。

\indent 2.本文中并未对大语言模型进行微调,而是直接使用了训练好的开源模型或GPT-3.5等闭源模型。
但是,大语言模型工具集成的一个重要研究方向就是如何通过微调让模型获得更好的工具调用效果。
现在有许多工作聚焦于微调规模较小的大语言模型,以提升其对工具任务的理解能力和工具方面的规划、推理能力。
经过大规模语料预训练的模型中具有许多工具调用的知识,在未来如何通过有监督的微调来更好挖掘大模型在工具
方面的潜力,并有效提升工具调用的效果,也是下一步需要研究的方向。

\indent 3.为了提升系统的易用性和用户体验,我们可以进一步优化系统的界面设计和
交互设计,使其更加直观和用户友好。同时,本系统仅在少量用户的情况下进行了测试,未来
可以对该系统的性能和稳定性进行持续改进,以确保在大规模用户同时使用的情况下依然能够保持高效
和稳定地提供服务。
