% !TEX root = ../main.tex

\chapter{总结与展望}

\indent 本章共包括两个部分,第一个部分是对全文工作内容进行总结和回顾,第二个部分就是对本工作的不足和局限性进行探讨,并对未来可能的研究探索方向进行探讨。

\section{工作总结}

本文从大语言模型在复杂任务场景中的应用挑战出发,提出了一套完整的解决方案,包括工具图谱构建、动态工具编排与调用方法,以及系统设计与实现,旨在探索如何通过大语言模型与知识图谱的结合实现高效的工具调用和任务解决方案。具体内容涵盖以下三方面:

\begin{enumerate}
    \item 本文提出了一种基于工具调用路径的工具图谱构建方法,系统性地完成了从数据筛选到图谱验证的全流程。通过提取高质量工具组和调用路径,设计了工具知识图谱的概念模型,以描述工具的层次关系与依赖逻辑,构建了工具图谱。该图谱为复杂任务中的工具选择与规划提供了重要支撑,显著提升了工具调用的准确性和效率。
    \item 为应对工具调用过程中的动态性和复杂依赖关系,本文设计了一种基于知识图谱的动态编排与调用方法。通过任务分解模块将复杂需求拆解为可执行的子任务,并结合深度优先搜索算法在图谱中动态规划工具路径,实现了工具调用的灵活性与精确性。引入记忆机制和自我反思机制,进一步提高了调用过程的稳定性和适应性,为多样化任务需求提供了高效解决方案。
    \item 本文基于知识图谱和大语言模型智能体机制,设计并实现了一个用户友好的API编排与调用系统。系统采用模块化架构,涵盖工具管理、任务规划与执行、以及智能问答功能,为用户提供了低门槛的交互体验。通过自然语言解析需求,系统能够自动完成API调用流程,实现了工具调用的自动化和可视化,展现了高效性与扩展性。
\end{enumerate}

\section{未来展望}

本文围绕着如何构建有效的API工具图谱、并将大量API工具集成到大语言模型中进行了一系列的探索,
然而本文工作仍面临着许多不足之处和可以持续探索的空间。
以下几点是未来可以持续探索和改进的方向。

\indent 1.本文中提出的基于工具调用路径数据构建工具图谱的方法,
在工具图谱的构建过程中,我们只考虑了工具之间的跳转关系,而没有考虑到工具之间的替换关系。
然而在现实工具调用场景中,当一个工具不稳定或者实效时,用户可能
需要选择其他在功能上有替换关系的工具进行代替。
如何在工具图谱中考虑更多的工具之间的关系,并挖掘更丰富的工具关系,是
未来值得探索的方向。

\indent 2.本文中并未对大语言模型进行微调,而是直接使用了训练好的开源模型或GPT-3.5等闭源模型。
但是,大语言模型工具集成的一个重要研究方向就是如何通过微调让模型获得更好的工具调用效果。
现在有许多工作聚焦于微调规模较小的大语言模型,以提升其对工具任务的理解能力和工具方面的规划、推理能力。
经过大规模语料预训练的模型中具有许多工具调用的知识,在未来如何通过有监督的微调来更好挖掘大模型在工具
方面的潜力,并有效提升工具调用的效果,也是下一步需要研究的方向。

\indent 3.为了提升系统的易用性和用户体验,我们可以进一步优化系统的界面设计和
交互设计,使其更加直观和用户友好。同时,本系统仅在少量用户的情况下进行了测试,未来
可以对该系统的性能和稳定性进行持续改进,以确保在大规模用户同时使用的情况下依然能够保持高效
和稳定地提供服务。
