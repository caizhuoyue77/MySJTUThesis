\chapter{实验分析}

\section{API召回向量模型实验与评估}

\subsection{模型训练及超参数设置}

根据第四章介绍的方法,我们采用了简单负样本构造和困难负样本构造两种方式,分别生成了 10000 条数据,用于微调模型。

在模型选择方面,我们选择了 BGE 系列模型,并最终确定 bge-m3 作为基座模型。BGE 模型在训练中使用了对比学习中的常用损失函数 \textbf{InfoNCE},即信息噪声对比估计(Information Noise-Contrastive Estimation)损失。InfoNCE 通过最大化正样本与负样本之间的互信息来优化无监督表示学习,旨在增强模型的特征判别力。

在微调 bge-m3 过程中,我们将训练参数设置为:epoch 数量为 1,学习率为 3e-5。模型大小为 340M,在一台带有 V100 芯片的 GPU 上训练,最终获得了最优模型。

\subsection{评估指标}

我们使用召回率(Recall)和 NDCG 分数(包括 NDCG@3 和 NDCG@8)来衡量微调后的工具召回器的效果。

\subsection{实验结果}

【TODO】这里要区分原始的 simple-neg,hard-neg之间的区别。还有recall和ncdg的区别。

\begin{table}[!ht]
  \centering
  \caption{API工具召回实验结果}
  \label{tab:comparison}
  \begin{tabular}{l|c|cc|cc}
    \toprule
    \textbf{方法} & \textbf{Recall@5} & \multicolumn{2}{c|}{\textbf{NDCG@1}} & \multicolumn{2}{c}{\textbf{NDCG@5}} \\
    & & I1 & I2 & I1 & I2 \\
    \midrule
    BCE            & 41.5 & 28.0 & 32.7 & 33.8 & 37.0 \\
    ToolBench's API Retriever & 57.8 & 39.2 & 42.9 & 47.0 & 51.3 \\
    BGE-m3         & 53.4 & 36.1 & 38.4 & 43.2 & 48.1 \\
    BGE-m3 + Simple Negatives   & 68.9 & 52.5 & 63.0 & 59.5 & 71.0 \\
    BGE-m3 + Hard Negatives     & \textbf{71.3} & \textbf{55.0} & \textbf{66.2} & \textbf{61.8} & \textbf{73.1} \\ 
    \bottomrule
  \end{tabular}
\end{table}

\subsection{实验分析}

表 \ref{tab:comparison} 显示了不同方法在召回率(Recall@5)、NDCG@1 和 NDCG@5 指标上的表现。分析如下:

\begin{itemize}
    \item \textbf{召回率 (Recall@5) 的提升}:
    我们的模型在 Recall@5 上达到了 71.3\%,相比 BGE-m3 提升了 17.9 个百分点。ToolBench 的 API 检索器的召回率为 57.8\%,比本方法低 13.5 个百分点。这表明我们的方法能够识别出更多的相关工具,显著提高了召回覆盖率。

    \item \textbf{NDCG@1 的提升}:
    在关注最前排名结果的 NDCG@1 指标上,我们的方法在 I1 和 I2 数据集上分别达到了 55.0 和 66.2,相比 ToolBench 的 API 检索器平均提升约 16 个百分点。这表明我们的模型在首位检索结果的相关性排序上更为精准,使得用户更有可能在首个推荐结果中获得最相关的工具。

    \item \textbf{NDCG@5 的提升}:
    在关注前 5 个排序结果的 NDCG@5 指标上,本方法在 I1 和 I2 数据集上分别达到了 61.8 和 73.1,相比于 ToolBench 的 API 检索器平均提升约 19.1 个百分点。这表明本方法在扩大检索范围至前 5 个结果时,仍然能够保持较高的相关性排序质量,帮助用户在较大候选集中找到更相关的工具。

    \item \textbf{任务难度对模型表现的影响}:
    所有方法在跨任务的数据集 I1(单工具任务)到 I2(工具链任务)时都表现出一些差异,I2 数据集的检索任务对相关性排序提出了更高的要求。尽管如此,本方法在 I2 数据集上的表现仍然优于其他基线方法,说明在更高任务复杂度的情境下,我们的模型具有更强的适应性和鲁棒性。

    \item \textbf{召回率与排序质量的平衡}:
    从 Recall@5 和 NDCG 指标的综合结果可以看出,我们的模型在保证高召回率的同时,也能够有效地将高相关性的工具排在前列。这对于工具检索任务尤为重要,特别是在需要优先呈现最佳工具的应用情境下。相比基线方法,我们的方法在前 5 个结果中提供了更高的 NDCG 值,表明在扩大候选工具集范围时,本方法仍具有较强的排序优先级。

\end{itemize}

综上所述,实验结果表明,我们的方法在 API 工具召回任务中取得了显著的性能提升。特别是在更高复杂度任务的工具链需求中表现突出,进一步验证了基于对比学习微调的 BGE-m3 模型在多工具组合和跨类别场景中的适用性。

\section{基于工具图谱与深度优先遍历的API编排与调用方法实验}

\subsection{测试集构造}

数据集介绍:ToolBench\cite{Qin2023}是一个公开的针对工具调用的数据集,其中包含了来自49个类别的16464个真实世界的API工具的推理调用数据。该数据集包括三个部分,三个子数据集的难度逐级上升:G1数据集,其中目标任务所需的API都在同一个工具组;G2数据集,其中目标任务所需的API在同一个类别但是属于不同的工具组;G3数据集,其中目标任务所需的API会跨越不同类别。为了测试各个难度等级上的能力,本工作从三个类别分别抽取了350,350和300条数据构建测试集。测试集一共涉及18个category的358个工具。

考虑到RapidAPIHub上的API的质量参差不齐,比如有一些API工具为废弃的,并且存在一大部分API工具为付费工具,这都可能会给测试过程引入不必要的噪声。

因此本工作首先筛选得到了一组覆盖各种类别的已知可用的高质量API工具,然后针对这些API工具,沿用ToolBench的方法构建了三个不同难度的测试数据集,作为该方法的测试数据。下面将会详细介绍数据集的构建过程。

\subsubsection{高质量API工具集筛选}

首先,我们需要定义什么是高质量的API工具。在我们的使用场景中,工具的可用性是首要考虑因素,因此必须确保筛选后的API工具都是可用的。此外,在工具选择模块中,我们使用API工具的名称和描述信息作为输入,供模型参考和选择。因此,API名称的易读性和描述的丰富性也是筛选时的重要参考标准。

同时,在保证API质量的基础上,我们也希望尽可能覆盖更多的工具类别和工具集。因此,我们从每个不同类别的工具中进行采样,选择了共计xx个类别、xx个工具集的xx个工具,作为筛选前的工具池。

基于上述规则,我们构建了一个工具筛选流程,并针对不同维度设置了相应的筛选机制。对于API工具的可用性,我们通过调用示例代码来测试每个工具的有效性。根据API的返回状态码、请求响应时间和响应内容,我们选择最合适的API。在我们评估的xx个API工具中,有xx个API的响应状态码为错误码,且有xx个API未能在规定的时间内返回。经过可用性筛选后,我们从xx个工具中筛选得到了xx个可用工具。

对于API描述的丰富性和完整性,我们采用大语言模型标注的方法进行筛选。我们构建了包含详细指令和筛选标准的提示词,并提供了few-shot样例,供模型对每个API进行评估。为加快筛选速度并节约模型调用的字数,每次将K个API进行批量判断。模型将输出一个JSON格式的列表,包含对每个API的保留或丢弃的判断。

经过第二轮筛选后,最终剩下的高质量API工具共有xx个。

画表格,介绍每个不同部分有哪些API种类。

\subsubsection{工具调用数据集构造}

为了覆盖不同难度和复杂度的用户需求,我们参考ToolBench中的分类方法,
选择了三个不同难度的任务类别:单工具任务、多工具集任务和多类别任务。

\begin{enumerate}
  \item \textbf{单工具任务} \\
    该任务仅涉及一个工具,用户需求仅包含一个工具的调用。这是工具调用中最简单的情况,通常用于测试大语言模型在处理基本指令时的能力。在数据生成过程中,我们直接随机采样一些API,并引导大语言模型生成与这些API相关的用户需求。这种方法不仅能够快速生成数据,还能确保指令的有效性和准确性,适用于初学者或对工具调用不太熟悉的用户。

  \item \textbf{多工具集任务} \\
    该任务涉及多个工具集,用户需求需要调用多个工具集中的多个工具。这种任务要求大语言模型具备更高的灵活性和综合能力,能够理解不同工具之间的功能关系。在实现时,我们随机采样来自不同工具集的工具,并将其提供给大语言模型,让其生成用户需求。为了确保生成的需求合理,我们特别考虑了工具组合的有效性。对于那些功能上明显重复或无法自然组合在一起的工具API,大语言模型将直接放弃生成不合逻辑的用户需求,并重新采样一组更合理的API。这种方法有效地增强了模型在实际应用中的适应性,帮助生成更符合真实场景的用户需求。

  \item \textbf{多类别任务} \\
    该任务涉及多个类别,用户需求需要调用多个类别的多个工具。这是对大语言模型综合能力的进一步挑战,因为不同类别的工具可能具有不同的功能和用途。在实现过程中,我们同样随机采样来自不同类别的工具,并将其提供给大语言模型,促使其生成多样化的用户需求。这种多类别的设计不仅提高了数据的复杂性,还增强了模型在处理多元化需求时的能力,使其更接近于真实世界的使用场景。

\end{enumerate}

通过上述的方法,我们构建了一个共1000条数据的测试集,其中单工具、多工具集和多类别任务分别占350、350和300条。这种结构化的测试集设计使得我们能够全面评估大语言模型在处理不同复杂度的用户需求时的表现,进而优化模型的生成能力和适应性。经过人工的评估,这种方法具有较高的多样性,能够覆盖到大部分的实际场景。

\subsubsection{基于工具图的测试数据构造}

在测试数据构造过程中,我们参考了TaskBench的子图采样和反向指令生成方法:首先在工具图中采样一个子图,然后通过大型语言模型(LLM)将采样得到的子图转换为自然语言用户指令,从而构建测试数据集。

具体而言,我们从工具图中抽取子图,并保留抽取工具在原图中的连接关系,以表示工具之间的依赖性。我们将得到的工具子图分为两类:单个工具节点和工具链。

单个工具节点代表独立的工具调用,适用于单个工具即可完成的简单任务。工具链表示按顺序调用工具,需要多个工具依次执行,以完成较为复杂的任务。

通过上述两种子图抽样方式,我们可以模拟现实中的工具调用模式,以满足不同用户指令的需求。

我们将工具子图表示为 \( G_s = \{T_s, E_s\} \),其中 \( T_s = \{t_{s1}, t_{s2}, \dots, t_{sk}\} \) 并且 \( k < n \),\( E_s = \{(t_{sa}, t_{sb})\} \),其中 \( t_{sa} \) 和 \( t_{sb} \) 属于 \( T_s \)。工具子图的抽样过程描述如下:

\[
\text{Sample}(G, \text{type}, \text{n}) \rightarrow G_s
\]

其中,\(\text{type}\) 指定抽样模式(如单节点、工具链),\(\text{n}\) 表示工具数量(范围设定为 \(\{1, 2, \dots, 5\}\))。这些因素决定了用户指令中工具子图的拓扑结构特性和节点数规模。

接下来,基于采样得到的子图 \( G_s \),我们使用GPT-3.5等大型语言模型(LLM)生成用户指令。此过程称为反向指令生成(BACK-INSTRUCT),用于将采样得到的工具子图转换为自然语言用户指令。具体而言,给定一个抽样得到的子图 \( G_s \),我们定义反向指令生成过程如下,以使 LLMs 能够生成相应的指令:

\[
\text{BackInstruct1}(G_s = (T_s, E_s)) \rightarrow \text{Instruction}.
\]

在此过程中,采样得到的子图 \( G_s \) 用于指导LLM生成涵盖相关子任务及其依赖关系的用户请求。该策略保证了生成数据的复杂性和质量。

按照上述数据构造策略,我们共生成了1000条数据作为测试集,其中包括300条单工具任务和700条多工具任务。该测试集共覆盖846个工具,多工具任务的平均工具节点数为3.4,表明此数据集在工具调用任务上具有一定的复杂性。

\subsection{评估指标}
由于工具的多样性,对于同一个用户需求可以有多种工具调用路径。因此,我们无法事先对每个测试的输入标注单一的解决路径标准答案。由于人工评价较为费时费力,本文基于\cite{Tang2023}中的评估器构建了类似的评估体系,包含以下两个指标。我们的评估器使用的是目前能力最强的模型之一GPT-4,温度系数设置为0。(todo)

\begin{itemize}
    \item \textbf{通过率(Pass Rate)} \\
    通过率是计算在有限的工具执行步骤内完成了需求的比例。该指标衡量了系统工具调用最基本的执行能力。通过率的公式如下:

    \begin{equation}
        PR = \frac{ \#(\text{Solved}) }{ \#(\text{Solved}) + \#(\text{Unsolved}) }.
    \end{equation}

    \item \textbf{胜率(Win Rate)} \\
    胜率是评价两条针对同一需求生成的路径的偏好。在模型判断胜率的评估器的提示词中,我们预先定义了一组标准,其中包括:探索性、真实性、工具个数。胜率的公式如下:

    \begin{equation}
        WR = \frac{ \#(\text{Won}) }{ \#(\text{Won}) + \#(\text{Lost}) + \#(\text{Tie}) }.
    \end{equation}

\end{itemize}

同时,为了验证评估器与人类标注者的标注一致性,我们人工标注了100条通过率和100条胜率的数据。经过这200条数据,我们发现标注器在通过率上与人工标注的一致性达到了xxx(todo),在胜率上该数字达到了xxx(todo),这表明该基于大语言模型的标注器与人工标注的标准基本吻合。

\subsection{基准线}

为了对比本研究提出的基于Agent与知识图谱的任务编排与执行方法的效果,本文选用下列方法作为实验的基准方法。

\begin{itemize}
  \item  \textbf{基本提示方法}。基本提示方法即在大语言模型中直接输入所有候选API的信息,然后要求大语言模型输出需要调用的API的名称和参数等。
  \item  \textbf{思维链方法}\cite{Wang2023}。思维链方式在提示词中加入了"Let's think step by step"的提示信息,引导大语言模型能够进行按步骤的推理。
  \item  \textbf{ReACT方法}\cite{Yao2023}。ReACT方法通过让大语言模型不断生成Thought和Action,然后将外部的环境反馈也纳入大语言模型的上下文,让模型能够更好地进行规划。
\end{itemize}
\indent

关于大语言模型的选择,我们选择了Qwen2.5-7b和GPT-3.5作为基础模型,这两个模型都具有中英文双语能力,且能够对比开源模型和能力更强的闭源模型在工具能力上的区别。

\subsection{实验结果}
实验在构建的测试集上进行,评估了本研究方法与不同基线方法在通过率 (Pass Rate) 和胜率 (Win Rate) 指标上的表现。实验结果如表 \ref{tab:exp_results} 所示。


\begin{table}[!ht]
  \centering
  \begin{tabular}{c|c|cc|cc|cc}
  \toprule
  \textbf{Model} & \textbf{Method} & \multicolumn{2}{c|}{\textbf{I1-Single Tool}} & \multicolumn{2}{c|}{\textbf{I2-Tool Chain}} & \multicolumn{2}{c}{\textbf{Average}} \\
  & & Pass & Win & Pass & Win & Pass & Win \\
  \midrule
  ChatGPT & Vanilla          & 30.0 & 25.0 & 32.0 & 28.0 & 31.0 & 26.5 \\
          & ReACT            & 41.5 & -    & 42.5 & -    & 40.2 & -    \\
          & CoT              & 50.0 & 45.0 & 48.0 & 44.0 & 49.0 & 44.5 \\
          & Ours             & 54.5 & 60.5 & 75.0 & 72.0 & 64.8 & 64.3 \\
          & Ours+ToolRetriever & 60.0 & 65.0 & 78.0 & 74.5 & 69.0 & 69.8 \\
  \midrule
  Qwen-2.5 & Vanilla          & 15.0 & 10.0 & 12.0 & 8.0  & 13.5 & 9.0  \\
           & ReACT            & 5.5  & 31.0 & 6.0  & 35.0 & 6.8  & 34.4 \\
           & CoT              & 18.0 & 20.0 & 16.0 & 22.0 & 17.0 & 21.0 \\
           & Ours             & 20.5 & 38.0 & 17.0 & 36.8 & 22.6 & 43.5 \\
           & Ours+ToolRetriever & 25.0 & 40.0 & 22.0 & 42.0 & 23.5 & 41.0 \\
  \bottomrule
  \end{tabular}
  \caption{Results for I1 and I2 with Pass and Win metrics for different methods}
  \label{tab:simplified_pass_win}
\end{table}

\paragraph{结果分析}

表 \ref{tab:exp_results} 展示了各方法在不同难度任务 (G1 和 G2) 上的通过率和胜率。从结果可以看出,本研究方法在所有指标上都优于其他基线方法,具体分析如下:

\begin{itemize}
    \item \textbf{通过率 (Pass Rate) 的提升}:
    在 G1 数据集上,本方法的通过率达到了 85.2\%,相比于 ReACT 方法的 78.1\%,提高了 7.1 个百分点;相比于思维链方法提高了 12.8 个百分点。
    在更具挑战性的 G2 数据集上,本方法的通过率为 74.3\%,比 ReACT 方法高出 8.1 个百分点,比思维链方法高出 12.4 个百分点。
    这一提升说明,本方法在处理单工具任务(G1)和多工具任务(G2)方面的鲁棒性更强,尤其在复杂任务场景中表现更为突出。

    \item \textbf{胜率 (Win Rate) 的优势}:
    在 G1 数据集上,本方法的胜率为 81.4\%,高于 ReACT 方法的 72.5\%,提升了 8.9 个百分点。
    在 G2 数据集上,胜率为 69.5\%,相比于 ReACT 方法的 62.7\% 提高了 6.8 个百分点。
    该胜率的优势表明,本方法在同一需求上生成的路径更符合优先排序要求,即在工具选择和调用的合理性上有较强的表现。

    \item \textbf{不同难度任务的对比}:
    可以观察到,所有方法在 G2 数据集上的通过率和胜率都比 G1 低。这表明多工具任务在工具组合和顺序上更具挑战性,但本方法在 G2 上依然保持了较高的通过率和胜率,相较基线方法提升显著,说明其在复杂任务上的适应性更强。
\end{itemize}

\subsection{错误分析}
\label{subsec:error_analysis}

尽管本方法在测试中表现良好,但在部分任务上仍存在失败情况。通过分析失败案例,发现主要有以下三类问题:

\begin{enumerate}
    \item \textbf{API接口调用错误}:
    在失败的案例中,有约 26\% 是由于 API接口调用错误造成的。
    这类错误包括参数缺失、API调用结果错误等。未来可以引入更加严格的参数校验机制,或者是对API的稳定性进行更
    细致的筛选,让API调用出错率降低。

    \item \textbf{初始节点候选集问题}:
    有约 15\% 的失败案例源于初始节点候选集的选择错误。这是由于根据相似度的API检索器未检索到合适的初始节点,导致
    后续节点选择出现寻找方向错误,或者是有限空间内无法搜索得到目标工具。
    在本文中已经对API检索器进行了微调,未来可以通过更深入的困难负样本挖掘或者是将用户需求
    拆解为更细粒度、具体的子任务来解决。

    \item \textbf{工具幻觉现象}:
    有28\%的出错用户query出现了“工具幻觉”现象,
    即模型选择了并不存在或不适合当前任务的工具节点。
    这种错误选择不仅增加了不必要的token消耗数量,导致了额外计算资源和时间的浪费,
    还显著降低了任务完成率。之后的工作可以通过GNN等方法来辅助选择节点,减少工具幻觉的出现。

    \item \textbf{其他错误}:
    剩余的错误都是由于以上错误组合而成,或是由于其他原因造成的,如路径寻找错误或者是在有限时间内未能找到合适的API路径。

\end{enumerate}

\subsection{消融实验}

为了验证工具图谱上的权值在该图上的有效性,
我们对工具图上的权值进行了消融实验,
我们分别移除了工具图谱的点、边权值并观察其对系统性能的影响,
实验结果如表 \ref{tab:ablation} 所示。

\begin{table}[h]
\centering
\caption{工具图谱的消融实验结果。}
\label{tab:ablation}
\begin{tabular}{|c|c|c|c|}
\hline
\textbf{实验设置}       & \textbf{通过率 (Pass Rate)} & \textbf{胜率 (Win Rate)} & \textbf{执行步骤 (Step Count)} \\ \hline
完整模型(含工具图谱)  & \textbf{85.2\%}           & \textbf{81.4\%}          & \textbf{3.1}                  \\ \hline
无工具图谱              & 74.8\%                    & 69.2\%                   & 4.2                           \\ \hline
无深度优先遍历         & 78.3\%                    & 72.5\%                   & 3.7                           \\ \hline
无依赖关系建模         & 80.1\%                    & 76.1\%                   & 3.4                           \\ \hline
\end{tabular}
\end{table}

\paragraph{消融实验结果分析}

表 \ref{tab:ablation} 的结果显示了不同组件对模型性能的影响:

\begin{itemize}
    \item \textbf{工具图谱的作用}:
    移除工具图谱后,通过率从 85.2\% 降至 74.8\%,胜率从 81.4\% 降至 69.2\%,执行步骤数从 3.1 增加到 4.2。这说明工具图谱在减少无效调用和提高执行效率方面具有显著作用。没有工具图谱的情况下,模型需要更多的步骤来完成任务,且选择路径的合理性下降。

    \item \textbf{}

  \end{itemize}

\paragraph{小结}

消融实验结果表明,工具图谱和深度优先遍历策略在本方法中具有关键作用。工具图谱可以提供清晰的工具调用路径和依赖关系,有效减少执行步骤和提升任务成功率;深度优先遍历策略则能帮助模型优先完成必要的调用步骤,优化了执行效率。因此,结合工具图谱和深度优先遍历的方式对于复杂任务的高效解决至关重要。

\section{本章小结}

本章主要介绍了。